\subsection{Descri��o e avalia��o hidrogeol�gica local especificando as caracter�sticas f�sicas dos aqu�feros e dos corpos h�dricos superficiais no trecho em que se inserem na �rea do empreendimento (larguras m�dia e m�xima, superf�cie)}

A �rea de influ�ncia do empreendimento encontra-se inserida em duas bacias
hidrogr�ficas: ao norte a bacia do rio Pirap� e ao sul a bacia do rio Iva�. Ambos os
rios (Pirap� e Iva�) des�guam no rio Paran�, sendo, portanto, 
constituintes da bacia do rio Paran�. O divisor de �guas entre as duas bacias
mencionadas posiciona-se numa faixa aproximadamente NW-SE ao longo da
rodovia BR 376 que liga o munic�pio de Maring� a Marialva, passando por Sarandi a
uma altitude m�dia de 550 m (a.n.m). De modo geral tratam-se de drenagens
pequenas com vaz�o m�dia raramente acima de 100 l$\cdot$s${^-1}$ mas que podem aumentar
para a jusante em dire��o �s drenagens principais, fora da �rea de influ�ncia.

Destaca-se que h� uma nascente e o Ribeir�o Aquidabam
que margeia a �rea do empreendimento.

%\clearpage

